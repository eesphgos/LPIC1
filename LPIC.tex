 \documentclass{article}


\title{CW GIT}

\date{2024 nov 18}

\author{Ehsan Esmaeili}

\newpage
\begin{document}

\section*{structure}
\subsection*{/}
\subsection*{/boot}kernel, initrd, bootloader and its config
\subsection*{/root}
\subsection*{/home}
\subsection*{/bin (binary)}user commands (general commands)
\subsection*{/sbin (system binary)}system commands (administration commands)
\subsection*{/lib (library)}shared libraries \& kernel modules
\subsection*{/opt (optional)}third-party applications
\subsection*{/tmp (temporary)}
\subsection*{/etc (etcetera)}host confiqurations files
\subsection*{/dev (device)}
\subsection*{/mnt (mount)}prepheral devices
\subsection*{/media}
\subsection*{/var (variable)}log server, print server, web server, mail server, cache server, client cache, a few applicarion caches,\ldots
\subsection*{/usr (user)}non-essential executable program struct of usr look like / and important next to /
\subsection*{/proc (process)}window to kernel
\subsection*{/sys (system)}window to kernel

\subsection{file}
every file have code named inode\#
\subsection{user}
every user have code named UID
\subsection{group}
every group have code named GID 
\subsection{processor}
every process have code named PID



\newpage
\section*{Commands}

\subsection*{ls}
ls command use to see all existing file in directory an you can make it optional wiht any switch
\subsubsection*{-a}
this switch show you all of file in directory even hidden file
\subsubsection*{-i}
show inode of files
\subsubsection*{-l}
this swithc show you file with more detale such as (type,user,group,size,modified date)


any type of file:

reqular file:-

directory: d

short cut (symbolic link, sym link): l

socket: s

pipe: p

special device-character: c

special device-block: b
\subsubsection*{*}
you can filter search by * \\
* mean zero to infinity character
\subsubsection*{?}
mean one character
\subsubsection*{[character]}
mean special character
\subsubsection*{[!character]}
mean not special character
\subsubsection*{[a-z]}
mean a range of character
\subsubsection*{[!a-z]}
mean not in range character
\subsubsection*{{arg1,arg2,\ldots}}
mean exacly same as every arguman
\subsection*{pwd}
show to you corrend directory address from root
\subsection*{whoami}
show your user
\subsection*{cd}
cd use for change directory
\subsubsection*{-}
change between two last address
\subsubsection*{absolute}
cd /fjj/ffs/s/f/f
\subsubsection*{relative}
cd-./dfff
cd-../dfdff (parent)
\subsection*{rm}
delete
!rm last rm command run
\subsubsection*{-r}
delete directory with content
\subsubsection*{-f}
no ask question
\subsection*{cp}
copy
\subsubsection*{-r}
copy directory
\subsection*{mkdir}
make directory
\subsubsection*{-p}
make any directory than not exist mkdir -p/s/sd/d
\subsection*{rmdir}
delete directory
\subsection*{clear}
clear terminal
\subsection*{mv}
move file
rename
move+rename
\subsection*{file}
show type of file
\subsection*{man command}
man show you manual of any command and tool to give help
\subsection*{hostname}
show hostname
\subsection*{init}
\subsubsection*{init 0}
shutdown os
\subsubsection*{init 6}
reboot os
\subsection*{echo}
write any thing to any file

echo \texttt{'}     \texttt{'} \texttt{>>} /  /file name

and show variable \texttt{-->} echo \&x

\subsubsection*{echo \$?}
if = 0 -\texttt{>} last command excute seccesfully
else last command excute failed
\subsubsection*{echo \$PATH}
path of commands
\subsubsection*{echo \$USER}
show user
\subsubsection*{echo \$UID}
user id
\subsubsection*{echo \$HOME}
\~:\(\)
\subsubsection*{echo \$SHELL}
default shell
\subsubsection*{echo \$OLDPWD}
old pwd
\subsection*{vi}
use vi to edit of creat text
\subsection*{passwd}
change password for users
\subsection*{exit}
exit from user
\subsection*{su \texttt{-} name}
switch user to name
\subsection*{ctrl (r)+fi,..,f6  or ctrl+alt+f1,..f6}
changing terminal
\subsection*{more}
in terminal you have no any thing like scroll so by more can see any thing withot removing

enter to go to the next line

space to go to the next page
\subsection*{less}
like more but you can go up and down again and agani and exit with q
\subsection*{nl}
count and writh lines
\subsection*{pipe: \texttt{|}}
give result of before function and get to next function

cat new.txt \texttt{|} nl \texttt{|} less 
\subsection*{semicolon: \texttt{;}}
if gone between to command commands are do like always
\subsection*{ampersand: \texttt{\&\&}}
if a command run succesfully them run next command
\subsection*{vertical bar: \texttt{||}}
like or do one of command

\#(echo in subshell;exit 4)\&\& echo OK \texttt{||} echo bad exit

in subshell

bad exit
\subsection*{touch}
make file

touch a1 a2 a3 b1 b2 b3 c1 c2 c3

touch {a,b,c}{a,2,3} `wildcard'
\subsection*{sort}
sort line of file (cat/tac)
\subsection*{which command}
tell me which path run `command'
\subsubsection*{-a}
tell me all address than run this command
\subsection*{whereis command}
give me all path that have `command'
\subsection*{info command}
information:\(\)
\subsection*{useradd/adduser}
add reqular user

hostname \texttt{>>/>}

date \texttt{>>/>}

dmesg \texttt{>>/>}
\subsection*{userdel}
delete user
\subsection*{qroupadd}
add qroup
\subsection*{cat}
show file text
\subsubsection*{-n}
like cat file \texttt{|} nl
\subsection*{alias}
save long command to variable
\subsubsection*{unalias}
delete alias
\subsection*{meta character}
\subsubsection*{*}
\subsubsection*{?}
\subsubsection*{[character]}
\subsubsection*{[!character]}
\subsubsection*{[a-z]}
\subsubsection*{[!a-z]}
\subsubsection*{{arg1,arg2,\dots}}
\subsection*{df}
show storage partition
\subsubsection*{-h}
show storage partition
\subsubsection*{-hT}
show type of partition
\subsubsection*{-i}
show inodo if partitions
\subsection*{lsblk}
show hard devision
\subsection*{fdisk}
make partition

\subsection*{mkfs}
formating a partition

\subsection*{mount}
mount a directory to hard
\subsubsection*{umount}
un mount a directory
\subsubsection*{-a}

mount every partition in fstab file that added new

\subsection*{history}
show command history
!5 run fifth command from top
!-5 run fufth command from down
!!/!-1 run last command
!p last command that start with p

history -d 149 delet line 149

history -c clear all history

history -w after -c
\subsection*{tune2fs}
about partition
\subsubsection*{-l}
make list propertis of a partition
\subsection*{grep}
filter a text by a keyword
\subsection*{mkswap}
make partition swap after choosing filesystem linux swap

mkswap /dev/name
\subsection*{swapon}
make swap on

swapon /dev/name
\subsection*{swapoff}
make swap of

\subsection*{uname -a}
kernel version

\subsection*{du -csh /addres}
list disk usage of addres physicaly

\subsection*{chmod}
change permission

with cahr num of | reference
\texttt{---------}

 u | g | o
 
||a||

chmode o+w file name

\subsection*{chown}
change owner
\subsubsection*{-r}
do to all child file
\subsubsection*{:}
chgrp = chmod \texttt{:} 
\subsection*{chgrp}
change group
\subsection*{newgrp}
change principle group
\subsection*{stat file}
status of file
\subsection*{umask}
set default permission but woek inversly

\subsection*{find}
find [addres] -[option] [variable]

find / -name `*.mp3'

find / -user ali

find / -qroup it

find / -perm 0400

find / -size +10M

find / -size -500M

find / -mtime -7

find / -atime -2

find / -ctime -14

find / -mmin -60

find / -amin -60

find / -cmin -30
\\
find / -size +10M -size -50M

find / -mtime +7 -mtime -14

find / -amin +30 -amin -60
\\
find / -type d -name test in double coutation

find / -type f -name
\\
find /tmp -name name -user ali -group it -perm 0400 -size +40M


2> /dev/nul in end command not show you error

exec rm -f {} \; at the end of file delete all founded or any command

find / -lname for sym link 

find / -samefile for link

find / -inum number for link by inod

\subsection*{expr}

do math equation

\subsection*{bash}
open new bash
\subsection*{ps}
show maked bash

\subsection*{mail}
make email and can send to another user and can be read by mail commands

male ali: mail to ali
\subsection*{reset}
reset the terminal and fix some bugs

\subsection*{gzip}
zip the file 

-d \texttt{||} gunzip to unziping

show with gzcat
\subsection*{bzip2}
compress better usualy
   
-d \texttt{||} bunzip2 to unziping

show with bzcat
\subsection*{tar (tape archiver)}

-f filename (file)

-c (creat)

-x (extract)

-t (test)

-C path (destionation)

-v (verbose)

tar -cf \@./gooni.tar \@./m*

-z (gzip, gunzip)

-j (bzip2, bunzip2)

tar -czf gooni.tar.gz  fileneme

tar -zxf gooni.tar.gzcat

tar -cjf gooni.tar.bz2 filename

tar -jxf gooni.tar.bz2

\subsection*{time command}

show time to excure totaly a command

\subsection*{wget link}

download from link

\subsection*{sleep}
sleep 5

dilay for five second

\subsection*{seq}

seq 1 10

write one to ten in order

\subsection*{}
%////////////////////////////////////////////////////////////////////////////////
\newpage
\section*{any thing}

\subsection*{/etc/qroup}
list of all group
\subsection*{/proc/swaps}
list of swaps
\subsection*{/proc/meminfo}
list of memory
\subsection*{/proc/cpuinfo}
information of cpu
\subsection*{/proc/version}
version of linux
\subsection*{/etc/passwd}
name of all user 
maked user exist an 1000 to up

\subsection*{/etc/fstab}

file of partition mount
\subsection*{/dev}
partition

/dev/null lool like black hole 


\subsubsection*{/dev/hda,b,c,\ldots IDE (PATA)}
\subsubsection*{/dev/sda,b,c,\ldots (SCSI)}
\subsubsection*{/dev/sda,b,c,\ldots (SATA)}
\subsubsection*{/dev/sda,b,c,\ldots cool disk}

\subsection*{file of variable}

/.bash-profile underline instead minus

/.profile 

/.bashrc

/.bash-logout

\subsection*{vi app}
basic text editor in any linux
\subsubsection*{i}
go to the insert mode
\subsubsection*{I}
like i just go to the first of line
\subsubsection*{shift+r}
go to replace mode repalce new to old char
\subsubsection*{esc key}
back to basic mode
\subsubsection*{:}
go to command line

q/q! \texttt{=>} quite from vi

w/w! \texttt{=>} save text /fours save

w /addres name\texttt{=>}  save as

n/n! \texttt{=>} open next file

e/e! /addres name\texttt{=>}  open file in addres/open force

r /addres name\texttt{=>  \ }  write file in this file

x/ZZ \texttt{=>} save and quite

every bash command \texttt{=>} \ldots

set nu \texttt{=>} set nummber to any line

nummber \texttt{=>} go to line number

\subsubsection*{v}
go to visual mode
\subsubsection*{o}
open new line and go to insert mode
\subsubsection*{O}
open new line before curser and go to insert mode

\subsubsection*{hjkl}
left down up right

3j = j+j+j

\subsubsection*{H,L,gg,G}

H go to first line of page

L go to last line of page

gg go to first of file

G go to last line of file
\subsubsection*{w,b}

w go to next word

b go to before word

\subsubsection*{ \^ 0 \$ } 

\^ go to first of first word of line

0 go to first of line

\$ go to last of line

\subsubsection*{ctrl+f,ctrl+b}

f go to forward page

b go to back page

\subsubsection*{x,X}

x delete from front of curser

X delet from back of curser

\subsubsection*{s,S}

s delete one char and go to insert mode

S delete one line and go to insert mode
\subsubsection*{dd}
cut total line

7dd cut 7 line
\subsubsection*{d num jahat}
cut nummber type jahat

\subsubsection*{c}
like d them go to insert mode
\subsubsection*{cc}
like dd then go to insert mode

\subsubsection*{y}%yank
copy to clip board
\subsubsection*{yy}
copy total line to clip board

\subsubsection*{p,P}

p past to right of curser

P past to left of curser

\subsubsection*{search}

/word

?word

n down

N up

\subsubsection*{u,ctrl+r,.}
undo

redo

repate last command
\subsection*{partitioning}

cable connection 

partitioning fdisk

chossing filesystem

format mkfs

mount mount!=unmount

to view df -h

\subsubsection*{partitioning method}
1- MBR Schene (MSDOS Style)\\
Master Boot Record ----\texttt{>} 512B
\newline
2- GPT Scheme\\
GUID Partition Table
\subsubsection*{MBR}
max 4

if primary = max 4 extended=n/a logical=n/a
if primary = 3 extended= max 1 logical= max 11
extended partion cant store data and contain logical partition

\subsubsection*{filesystem}

native | ext2,3,4,btrfs,reiserfs\\
non-native if 

    with modules | XFS,ZFS

    withoun module | APFS,NTFS
\\
cross-platfor | FAT12,16,32,iso9660,CDFS (joliet)
\\
small partition \texttt{<=} 2TB \texttt{<} large partition

\subsection*{virtual memory}
windos page file
\\
linux swap
\\
with LRU algorithm (least Recently Used)
\subsubsection*{free -h}
show your swap
\subsection*{dd if=/dev/zero of=/root/myswap bs=500M count=1}
make 500M free of / partition and named myswap

then with mkswap /root/myswap and swapon /root/myswap add it to main swam

\subsection*{link}
\subsubsection*{hard link / link}

make with ln filename linkname

like backup

look like pointer to inode

not work in diffrent partititon
\subsubsection*{symbolic link / sym link}
make with ln -s filename linkname

this link look like shortcut 


\subsection*{variable}
\subsubsection*{system var: usualy capital}
two module

local var | v = any

environment var | export v = any

variable in linux:

question sigh (last command)

PATH (addresss of commands)

USER (user name that login)

HOME (addres of home)

SHELL (addres of bash)

PWD (save addres of current directory)

OLDPWD () work with cd \texttt{-}

HISTSIZE (500 last command)

MAIL (address of file than save email of user)

HOSTNAME

LS (underscore sign) COLORS

\$ current shell PID

PPID curren shell parent PID

PS1 file of text than in terminal showed before every commands
\\
ps1 code:

u user

h computer name

H hostname

d date

t time

w totla current addres

W last current directory

n new line

\$ user type sign

e color code

\dots\dots\dots\dots\dots\dots\dots

env list of total environment variable

\subsubsection*{user defined var: usualy small}

\subsection*{Man Pages:}

1.user commands (general commands)

2.system calls

3.library class

4.

5.config files

6.games
7.miscellaneous

8.system commands (administration command)

9.kernel routines

all in /usr/share/man/\dots

whatis fdisk

apropos partition

fdisk --help (wiht all commands)
\subsection*{text processing}
work with delimiter

set delemiter 

cut -d double coutation and delemiter -f 1 toby- 3 /etc/passwd > u3
cut -d double coutation and one char -f 1,3 /etc/passwd

-f feild

head file

heat default=10 line

head -5 file name

tail like head

tale or head \texttt{-}n

expand command change tab to space default 8 and you can use -t 5 to five space

unexpand change space to tab

od -tc/tc file name show tab or spase and totaly char

sort

-r reverce

-n sort numeric

-k 2 with feild 2

wc line/word/char number

-l -w -m -c --byte

uniq filename del repared line

-d print repeated line

-D print repeated line and count of repeated

-c count of repeat

join file1 file2 one colum most be same an another file

join -1 3 -2 4 file1 file2 join with another column 3 in file one and 4 in file two

paste write file one them filetwo in row

-d and double coutation can use delemiter

split -num filename name of new file 

default 1000line

split -b 4.7G file name file

cat file1 file2 show file2 after file1

rev recerse in every line

tac revers in total of file

tr a-z A-Z if a them A just srandard entry

recive operator <

xargs in standard entry and write all of file in one line

ls /home 1>filename 2>filename/\&1 pipe one

tee command like three way on input two out put that on of always file amd another is terminal of pipe

tee -a append text

grep anything

grep -v none have 

grep -i case insensitive

grep -n line nubber

or sign cahr at the first of line bee char

\& char at teh last of line be char

\ <word\> just word total in coutation

[abcd] [a-b] 

or sign in bracet mean not these char

.show lines than not null or just enter

\subsection*{Processes}
spawn

-PID

-lifetime normal exit or abnormal exit

-UID

-GID

-Parent Processes

-workingdirectory

-environment

-\dots
\\
ps show process

-A (all)

-e (all)

-a (attached processes)

-f (forest mode) ps tree -h/-p

-w (wide output format)

-l (long format)

-u (user format)

-x (background processes)

-C command (special commands)

-U username (User command)

most useful

ps -aux

ps -ef

ps -el 

top like taskmanager and put space lead to refres bar (default 3s)

uptime show some detale like time uptime upusers lead avg

d change delay 

k kill a process

kill -l  list of signals

SIGHUP reload 1

SIGINT interrupt,cancle 2          ctrl+c

SIGKILL kill 9                     

SIGTERM terminate 15

SIGCONT continue,unfreeze,unpause 18

SIGTSTP stop,freeze,pause 20       ctrl+z

kill -15 PID\#

kill PID\# default -15

kill -9 PID PID PID PID PID

killal vi

NI (nice of process) priority of process between -20 to +19
and +19 wors priority and -20 is the best
just super user can lower priority

nice -4 vi myfile

nice --vi myfile

renice in during the process change priority

renice 9 PID

renice -9 PID

tail -f show last of file and live it

du -csh /var

du -cs /var

du -sh /var

du -ch /var

jobs -l running command

fg num run one of those to foregrand

bg like fg but run in back ground
 
\& at the last of command lead to command run in background

nohub anycommand \& keep running after logout 

\subsection*{useradd}
swithc
 
-d new home dir addres

-m make address than is up in disk

-g name of principle user group 

-G name of subordiner user qrouop 

-s address of excuring default shell 

-u user UID

-c command of discription

usermod -L name

usermod -U name

passwd -l name

passwd -u name

cshs to chang shell  

\subsection*{login}
l

su \texttt{-} ali login mode

su ali non-login mode

/etc/skel/ files are coped to any when an account maked

/.bash (underline) profile 1 3 no excute in none login mode

/.bashrc 2 2

/etc/bashrc 3 1

number is order of read and do

\dots\dots\dots\dots

/.bash (underline) logout 

su/su -empty switch user to root

visudo like vi /etc/sudoer

visudo of vi /etc/sudoer in this file can change accesblity of another user
and user than in sudoer file can be like root and log be save its activity

in this file \% before name mean its group not user

in some linux root have no password
with sudo su \texttt{-} root and them change root password

useradd -u 500 newusr if before exsist an user with UID 500 and we delete it
and want to get this user to new user use this command

userdel -r name delete user and home directory

groupdel name

pwuncomv deactivate shadowing grpunconv

pwconv activate shadowing grpunconv

\subsection*{boot initialisation and shutdown}

\hbox{BIOS} (Basic input/output System)

hardware clock (CMOS battery)

software clock

CHS (sylinder/head/sector)
LBA (logical block addressing)

EFI

UEFI

\hbox{MBR} (master boot record)

512 B

446 B (bootloader infrmation)

64 (partition tavle informatin)

2B (MBR validation check)

most famous bootloader in linux

\hbox{1. LILO (linux loader) lili /etc/lilo.conf
}
\hbox{2. Grand unified boorloader (GRUB legacy) grub /boot/grub/menu.lst /boot/grub/grub.conf
/etc/grub.conf  FHS  LSB}
\hbox{3. GRUB2 grub  /boot/grub/grub.cfg /boot/grub2/grub.cfg /etc/\dots}

initrd is a file than kernel need to it to prevent  detectivr loop

\hbox{bootloader}
\hbox{kernel }
\hbox{init}
\hbox{runlevel}

init and telinit to change runlevel

init 0 shotdwon

inti 1 single user mode

init 2 multiuser -nfs -x11

inti 3 multiuser +nfs -x11

init 4 unused

init 5 multiuser +nfs +x11

init 6 reboot 

we cant use init 0 or 6 in default 

usualy default is 5 or 3

/etc/inittab

runlevel command show before and now runlevel
N mean shud down and s mean single user mode

unused runlevel work like before runlevel init 4 = init 3

startx up graphic sevise manual withot use init 5

service web ssh database dns proxy/cache 

applications apache openssh-server vsftb mysql bind squid

daemon httpd sshd vsftpd mysqld named squid

name of script like daemon

/etc/init.d/sshd stop|start|status|restart|reload|

/etc/rc0.d

/etc/rc1.d

/etc/rc2.d

/etc/rc3.d

/etc/rc4.d

/etc/rc5.d

/etc/rc6.d

S{n}{n}sshd to start

K{n}{n}sshd to stop

for start s00 to  s99 in priority in /etc/rc{n}

also for end

\hbox{redhat base:}
1. LSB\@: /etc/init.d/sshd stop|start|status|restart|reload|\dots

2\@. service sshd stop|\dots

3\@. chkconfig

4\@. systemstl

\hbox{debian base}

1. LSB

2\@. service

3\@. update-rc.d

4\@. systemctl

example for 3

update-rc.d crond default

update-rc -f dovecot remove

update-rc -f dovecot stop 24 2 3 4 5

\subsection*{service managment}

-System V (SssysVInit) --> init (centoOS 5, older)

service sshd stop

chkconfig || update-rc.d

init 0

-upstart --> init (CentoOS 6)

initctl stop sshd

chkconfig || update-rc.d

init 0

-systemd --> (CentOS 7, newer)

systemctl stop sshd ----> at moment

systemctl start sshd ---> at moment

systemctl disable sshd -> next boot and always

systemctl enable sshd---> next boot and always

systemctl poweroff

-shutdown command

shutdown -h now

shutdown -h 120

shutdown -r 10


-h (shutdown, halt)

-r (reboot)

-c (cancle)

-f (fast boot)

-F (force filesystem check)

\subsection*{package managment}

1\@. redhat rpm *.rpm

2\@. debian dpkg *.deb

a-b.c.d-e.f.rpm

a=package name

b=version

c=major release (major revision)

d=minor release (minor revision)

e=build number

f=arch. --> x89, i386, i486, i586, i686
  --> 32bit
--> x86{underline}64, amd64 --> 64bit

--> sparc, 

--> PowrPC, 

--> arm, 

--> \dots, 

--> all, noarc (suit for any device)

\hbox{rpm}

-i (install)

-u (update)

-U (install/update)

    -v (verbose)

    -h (hash)

-e (erase)

-q (query)

    -a (list all package names installed on this system)

    -f (file)

    -l (list all package files)

    -c (list all package config files)

which chmod

rpm -qf /usr/bin/chmod

\hbox{dpkg}

-i (install) = rmp -i

-r (remove)  = rmp -e

-s (status, query)  = rmp -q

-S (search) = rmp -qf

-l (list all installed package) = rmp -qa

-L (list all package files) = rmp -ql

\hbox{wrapper}

1.yum (redhat base)

2.apt (debian base)

\hbox{yum}
(yellowdog updater madified)

/etc/yum.conf

/etc/yum.repos.d/---.repos

yum update

yum update appname

yum install appname

yum remove appname
\hbox{apt}
(advanced package tools)

/etc/apt/sources.list

/etc/apt/sources.list.d/official-package-repositories.list

apt -get update

apt -get upgrade

apt -get upgrade appname

apt -get install appname

apt -get remove appname

-get is optional

\subsection*{shell:}

sh, csh, ksh, tcsh, zsh, \dots bash, \dots

you shold write your command in a text and run it with
bash command 

>>echo clear >> a

>>bash a

run file as alone command most use addres 

>>addres/a to run

or

vi \texttt{.}bash{underline}profile

and change PATH (add address of this file)

better named file than have \texttt{.}sh at end

\# to comment

\#! shibang addres runner of command 
so \#! /bin/bash

>>which top
>>ls -l /usr/bin/top

>>ls -l `which top\texttt{`}

>>ls -l \$(which top)

these command are equal to each other

aotumatic run in linux in cron service

crontab

-l list

-r remover

-e edit

min hrs dom moy dow cmd these caloumn ser the time of script

15  8   *   *   *   /root/class/myscript.sh  these line run than exploitat every day in 8:15

15  8\texttt{-}10   *   *   *   /root/class/myscript.sh

15  8,17  *   *   *   /root/class/myscript.sh


how work with crone

cronetab -e

write => 53 10 * * * /root/class.myscript.shadowing

\subsection*{networt}

\hbox{OSI}
client           server
1.Application    A
2.Presentation   P
3.Session        S
4.Transport      T
5.Network        N
6.Data link      D 
7.Phisical      P

\hbox{physical layer}

Wireless   wlan0

Ethernet eth0

priority with Ethernet

\hbox{naming approaches}
1-Hostname = computerP{underline}name.domain{underline}name

www.google.com

mail.yahoo.com

kashani.lpir.org

(org, com, ir) TLD (top level domain)

(google, yahoo, lpir) SLD (second level domain)

www, mail sub\texttt{-}domain name

change hostname in /etc/hostname

2-Physycal addr (mac addr)

dhclient (IP, Gateway, DNS)

3-IP Addr

ifconfig eth0 192.168.10.11

default netmasks 

--> A\@:255.0.0.0

--> B\@:255.255.0.0

--> C\@:255.255.255.0

ifconfig eth0 192.168.10.11 netmask

11111111.11111111.11111111.11111111.00000000

debian base: /etc/network/interface

redhat base: /etc/sysconfig/network-scripts/ifcfg-eth0

\hbox{command}
ifconfig

ip addr show  | ip a

\dots

ifconfig eth0 192.168.10.11

ip addr add 192.168.10.11 dev eth0

\dots

ifconfig eth0 down | ifdown eth0

ip link set eth0 down

\dots

ifconfig eth0 up | ifup eth0

ip link set eth0 up

\dots

route -n

ip route show | ip r

\dots

route add default gw 192.168.10.1

ip route add default via 192.168.10.1

\hbox{ssh (22)}

client ------------> server A --------------> server B

openssh-clients      openssh-server (22)
                     openssh-clients        openssh-server (22)

1.remote console

ssh username@IP

2.secure copy

scp usesrname@IP:path username@IP:path

for each user that i write in command in terminal can write no username and just write addres


\end{document}  