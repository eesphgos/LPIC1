\documentclass[12pt, a4paper]{article}
\usepackage[a4paper, margin=1in]{geometry}
\usepackage[utf8]{inputenc}
% \usepackage[farsi]{babel}
% \usepackage{fontspec}
\usepackage{xcolor}
\usepackage{titlesec}
\usepackage{fancyhdr}
\usepackage{enumitem}
\usepackage{listings}
\usepackage{amsmath}
\usepackage{graphicx}
\usepackage{tcolorbox}
\usepackage{hyperref}

% تنظیم فونت فارسی
% \setmainfont{Arial}
% \newfontfamily\farsifont{Arial}

% تنظیمات صفحه
\pagestyle{fancy}
\fancyhf{}
\fancyhead[L]{\textbf{LPIC-1 Study Notes}}
\fancyhead[R]{\textbf{Ehsan Esmaeili}}
\fancyfoot[C]{\thepage}
\renewcommand{\headrulewidth}{0.4pt}
\renewcommand{\footrulewidth}{0.4pt}

% تنظیمات بخش‌ها
\titleformat{\section}
{\normalfont\Large\bfseries\color{blue!70!black}}
{}{0pt}{}

\titleformat{\subsection}
{\normalfont\large\bfseries\color{green!50!black}}
{}{0pt}{}

\titleformat{\subsubsection}
{\normalfont\normalsize\bfseries\color{orange!70!black}}
{}{0pt}{}

% تنظیمات لیست
\setlist[itemize]{leftmargin=*, label=\textbullet}
\setlist[enumerate]{leftmargin=*, label=\arabic*.}

% تنظیمات کد
\lstset{
    basicstyle=\ttfamily\small,
    breaklines=true,
    frame=single,
    backgroundcolor=\color{gray!5},
    tabsize=2
}

% عنوان
\title{
    \textbf{\Huge LPIC-1: Linux Professional Institute Certification} \\
    \vspace{0.5cm}
    \Large Comprehensive Study Notes \\
    \vspace{0.2cm}
    \normalsize Version 1.0
}
\author{\textbf{Ehsan Esmaeili}}
\date{\textbf{November 18, 2024}}

\begin{document}

\maketitle

\begin{center}
    \rule{0.8\textwidth}{0.5pt}
\end{center}

\begin{tcolorbox}[colback=blue!5!white, colframe=blue!75!black, title=\textbf{Disclaimer}]
    These notes are prepared for LPIC-1 certification exam preparation. They cover essential Linux concepts, commands, and system administration topics.
\end{tcolorbox}

\tableofcontents
\newpage

\section{Linux Filesystem Hierarchy Standard (FHS)}
\label{sec:structure}

\begin{tcolorbox}[colback=green!5!white, colframe=green!75!black, title=\textbf{Key Points}]
    Understanding the Linux directory structure is fundamental for system administration.
\end{tcolorbox}

\subsection{Directory Structure}
\begin{itemize}
    \item \textbf{/} - Root directory
    \item \textbf{/boot} - Kernel, initrd, bootloader and its configuration
    \item \textbf{/root} - Root user's home directory
    \item \textbf{/home} - Users' home directories
    \item \textbf{/bin} (binary) - User commands (general commands)
    \item \textbf{/sbin} (system binary) - System commands (administration commands)
    \item \textbf{/lib} (library) - Shared libraries \& kernel modules
    \item \textbf{/opt} (optional) - Third-party applications
    \item \textbf{/tmp} (temporary) - Temporary files
    \item \textbf{/etc} (etcetera) - Host configuration files
    \item \textbf{/dev} (device) - Device files
    \item \textbf{/mnt} (mount) - Mount point for peripheral devices
    \item \textbf{/media} - Mount point for removable media
    \item \textbf{/var} (variable) - Variable data (logs, spool, cache, etc.)
    \item \textbf{/usr} (user) - Non-essential executable programs
    \item \textbf{/proc} (process) - Virtual filesystem providing process and kernel information
    \item \textbf{/sys} (system) - Virtual filesystem providing system information
\end{itemize}

\subsection{System Identifiers}
\begin{itemize}
    \item \textbf{File}: Every file has a unique identifier called \texttt{inode\#}
    \item \textbf{User}: Every user has a unique identifier called \texttt{UID}
    \item \textbf{Group}: Every group has a unique identifier called \texttt{GID}
    \item \textbf{Process}: Every process has a unique identifier called \texttt{PID}
\end{itemize}

\newpage
\section{Basic Linux Commands}
\label{sec:commands}

\subsection{File Listing - \texttt{ls}}
The \texttt{ls} command lists files and directories.

\begin{lstlisting}[language=bash, caption=ls command examples]
ls          # Basic listing
ls -a       # Show all files including hidden
ls -l       # Long listing with details
ls -i       # Show inode numbers
ls -la      # Combine options
\end{lstlisting}

\subsubsection*{File Types in Detailed Listing}
\begin{itemize}
    \item \texttt{-} - Regular file
    \item \texttt{d} - Directory
    \item \texttt{l} - Symbolic link
    \item \texttt{s} - Socket
    \item \texttt{p} - Pipe
    \item \texttt{c} - Character device
    \item \texttt{b} - Block device
\end{itemize}

\subsubsection*{Wildcards and Pattern Matching}
\begin{lstlisting}[language=bash]
ls *.txt           # All .txt files
ls file?.txt       # file1.txt, file2.txt, etc.
ls [abc]*          # Files starting with a, b, or c
ls [!abc]*         # Files not starting with a, b, or c
ls {file1,file2}   # Specific files
ls file[1-5].txt   # Files file1.txt through file5.txt
\end{lstlisting}

\subsection{Navigation Commands}
\begin{itemize}
    \item \textbf{\texttt{pwd}} - Print working directory
    \item \textbf{\texttt{whoami}} - Display current username
    \item \textbf{\texttt{cd}} - Change directory
    \begin{itemize}
        \item \texttt{cd -} - Switch between two last directories
        \item \texttt{cd /path} - Absolute path
        \item \texttt{cd ./dir} - Relative path
        \item \texttt{cd ../dir} - Parent directory
    \end{itemize}
\end{itemize}

\subsection{File Operations}
\begin{itemize}
    \item \textbf{\texttt{rm}} - Remove files
    \begin{itemize}
        \item \texttt{rm -r} - Remove directories recursively
        \item \texttt{rm -f} - Force removal without confirmation
    \end{itemize}
    \item \textbf{\texttt{cp}} - Copy files
    \begin{itemize}
        \item \texttt{cp -r} - Copy directories recursively
    \end{itemize}
    \item \textbf{\texttt{mv}} - Move/rename files
    \item \textbf{\texttt{mkdir}} - Create directories
    \begin{itemize}
        \item \texttt{mkdir -p} - Create parent directories if needed
    \end{itemize}
    \item \textbf{\texttt{rmdir}} - Remove empty directories
    \item \textbf{\texttt{touch}} - Create empty files or update timestamps
\end{itemize}

\subsection{File Examination}
\begin{itemize}
    \item \textbf{\texttt{file}} - Determine file type
    \item \textbf{\texttt{cat}} - Concatenate and display files
    \begin{itemize}
        \item \texttt{cat -n} - Number all output lines
    \end{itemize}
    \item \textbf{\texttt{more}} - View file contents page by page
    \item \textbf{\texttt{less}} - Improved version of \texttt{more}
    \item \textbf{\texttt{nl}} - Number lines of files
\end{itemize}

\subsection{System Information}
\begin{itemize}
    \item \textbf{\texttt{hostname}} - Show or set system hostname
    \item \textbf{\texttt{uname -a}} - Show all system information
    \item \textbf{\texttt{df}} - Display disk space usage
    \begin{itemize}
        \item \texttt{df -h} - Human readable format
        \item \texttt{df -hT} - Show with filesystem type
        \item \texttt{df -i} - Show inode information
    \end{itemize}
    \item \textbf{\texttt{du}} - Estimate file space usage
    \begin{itemize}
        \item \texttt{du -sh} - Summary in human readable format
        \item \texttt{du -csh} - Total summary
    \end{itemize}
    \item \textbf{\texttt{lsblk}} - List block devices
    \item \textbf{\texttt{free -h}} - Display memory usage
\end{itemize}

\subsection{Text Processing}
\begin{itemize}
    \item \textbf{\texttt{grep}} - Search text using patterns
    \begin{itemize}
        \item \texttt{grep -i} - Case insensitive
        \item \texttt{grep -v} - Invert match
        \item \texttt{grep -n} - Show line numbers
    \end{itemize}
    \item \textbf{\texttt{cut}} - Remove sections from lines
    \item \textbf{\texttt{sort}} - Sort lines of text
    \item \textbf{\texttt{uniq}} - Report or omit repeated lines
    \item \textbf{\texttt{wc}} - Word count
    \begin{itemize}
        \item \texttt{wc -l} - Count lines
        \item \texttt{wc -w} - Count words
        \item \texttt{wc -c} - Count bytes
    \end{itemize}
\end{itemize}

\subsection{Permission Management}
\begin{itemize}
    \item \textbf{\texttt{chmod}} - Change file permissions
    \begin{itemize}
        \item \texttt{chmod u+w file} - Add write permission for user
        \item \texttt{chmod 755 file} - Numeric permission setting
    \end{itemize}
    \item \textbf{\texttt{chown}} - Change file owner
    \begin{itemize}
        \item \texttt{chown -r} - Recursive ownership change
        \item \texttt{chown user:group file} - Change both owner and group
    \end{itemize}
    \item \textbf{\texttt{chgrp}} - Change file group
    \item \textbf{\texttt{umask}} - Set default file permissions
    \item \textbf{\texttt{stat}} - Display file status
\end{itemize}

\subsection{Process Management}
\begin{itemize}
    \item \textbf{\texttt{ps}} - Report process status
    \begin{itemize}
        \item \texttt{ps aux} - Detailed process information
        \item \texttt{ps -ef} - Full format listing
        \item \texttt{ps -el} - Long format
    \end{itemize}
    \item \textbf{\texttt{top}} - Dynamic real-time view of processes
    \item \textbf{\texttt{kill}} - Send signals to processes
    \begin{itemize}
        \item \texttt{kill -9 PID} - Force kill process
        \item \texttt{kill -15 PID} - Terminate gracefully
    \end{itemize}
    \item \textbf{\texttt{nice}} - Run with modified scheduling priority
    \item \textbf{\texttt{renice}} - Alter priority of running process
\end{itemize}

\subsection{Package Management}
\subsubsection{RPM-based Systems (RedHat/CentOS/Fedora)}
\begin{itemize}
    \item \texttt{rpm -i package.rpm} - Install package
    \item \texttt{rpm -e package} - Remove package
    \item \texttt{rpm -q package} - Query package
    \item \texttt{rpm -qa} - List all installed packages
\end{itemize}

\subsubsection{APT-based Systems (Debian/Ubuntu)}
\begin{itemize}
    \item \texttt{apt install package} - Install package
    \item \texttt{apt remove package} - Remove package
    \item \texttt{apt update} - Update package list
    \item \texttt{apt upgrade} - Upgrade packages
\end{itemize}

\newpage
\section{System Configuration Files}
\label{sec:configfiles}

\begin{tcolorbox}[colback=orange!5!white, colframe=orange!75!black, title=\textbf{Important Configuration Files}]
    These files are crucial for system administration and troubleshooting.
\end{tcolorbox}

\begin{itemize}
    \item \textbf{/etc/passwd} - User account information
    \item \textbf{/etc/group} - Group information
    \item \textbf{/etc/shadow} - Secure user password information
    \item \textbf{/etc/fstab} - Filesystem table
    \item \textbf{/etc/hostname} - System hostname
    \item \textbf{/etc/hosts} - Static hostname lookup table
    \item \textbf{/etc/resolv.conf} - DNS resolver configuration
    \item \textbf{/etc/sudoers} - Sudo configuration
    \item \textbf{/etc/ssh/sshd\_config} - SSH server configuration
    \item \textbf{/etc/crontab} - System cron jobs
\end{itemize}

\subsection{User Environment Files}
\begin{itemize}
    \item \textbf{\textasciitilde/.bash\_profile} - Login initialization
    \item \textbf{\textasciitilde/.bashrc} - Non-login shell initialization
    \item \textbf{\textasciitilde/.profile} - Default profile
    \item \textbf{\textasciitilde/.bash\_logout} - Logout actions
\end{itemize}

\subsection{System Information Files}
\begin{itemize}
    \item \textbf{/proc/cpuinfo} - CPU information
    \item \textbf{/proc/meminfo} - Memory information
    \item \textbf{/proc/version} - Linux version
    \item \textbf{/proc/swaps} - Swap information
    \item \textbf{/proc/loadavg} - System load average
\end{itemize}

\newpage
\section{Network Configuration}
\label{sec:network}

\subsection{Network Commands}
\begin{itemize}
    \item \textbf{\texttt{ifconfig}} - Configure network interfaces
    \item \textbf{\texttt{ip addr show}} - Show IP addresses
    \item \textbf{\texttt{route -n}} - Display routing table
    \item \textbf{\texttt{ping}} - Test network connectivity
    \item \textbf{\texttt{netstat}} - Network statistics
    \item \textbf{\texttt{ss}} - Socket statistics
    \item \textbf{\texttt{hostname}} - Show or set hostname
\end{itemize}

\subsection{SSH Configuration}
\begin{itemize}
    \item \textbf{Connect to remote server}: \texttt{ssh user@hostname}
    \item \textbf{Copy files securely}: \texttt{scp file user@hostname:path}
    \item \textbf{SSH configuration file}: \texttt{/etc/ssh/sshd\_config}
\end{itemize}

\subsection{Network Configuration Files}
\begin{itemize}
    \item \textbf{Debian/Ubuntu}: \texttt{/etc/network/interfaces}
    \item \textbf{RedHat/CentOS}: \texttt{/etc/sysconfig/network-scripts/ifcfg-*}
    \item \textbf{DNS Configuration}: \texttt{/etc/resolv.conf}
    \item \textbf{Hosts file}: \texttt{/etc/hosts}
\end{itemize}

\newpage
\section{Disk Management}
\label{sec:diskmgmt}

\subsection{Partition Management}
\begin{itemize}
    \item \textbf{\texttt{fdisk}} - Partition table manipulator
    \item \textbf{\texttt{parted}} - Partition manipulation program
    \item \textbf{\texttt{gdisk}} - GPT fdisk
\end{itemize}

\subsection{Filesystem Operations}
\begin{itemize}
    \item \textbf{\texttt{mkfs}} - Build a filesystem
    \begin{itemize}
        \item \texttt{mkfs.ext4} - Create ext4 filesystem
        \item \texttt{mkfs.xfs} - Create XFS filesystem
    \end{itemize}
    \item \textbf{\texttt{mount}} - Mount filesystem
    \item \textbf{\texttt{umount}} - Unmount filesystem
    \item \textbf{\texttt{fsck}} - Check and repair filesystem
\end{itemize}

\subsection{Swap Management}
\begin{itemize}
    \item \textbf{\texttt{mkswap}} - Set up a Linux swap area
    \item \textbf{\texttt{swapon}} - Enable swapping
    \item \textbf{\texttt{swapoff}} - Disable swapping
\end{itemize}

\newpage
\section{System Services and Runlevels}
\label{sec:services}

\subsection{Systemd Service Management}
\begin{itemize}
    \item \texttt{systemctl start service} - Start a service
    \item \texttt{systemctl stop service} - Stop a service
    \item \texttt{systemctl restart service} - Restart a service
    \item \texttt{systemctl enable service} - Enable service at boot
    \item \texttt{systemctl disable service} - Disable service at boot
    \item \texttt{systemctl status service} - Check service status
\end{itemize}

\subsection{Runlevels}
\begin{itemize}
    \item \textbf{0} - Halt
    \item \textbf{1} - Single user mode
    \item \textbf{2} - Multi-user without NFS
    \item \textbf{3} - Full multi-user mode
    \item \textbf{4} - Unused
    \item \textbf{5} - Graphical mode
    \item \textbf{6} - Reboot
\end{itemize}

\subsection{Service Control Commands}
\begin{itemize}
    \item \textbf{\texttt{service}} - Run a System V init script
    \item \textbf{\texttt{chkconfig}} - Update runlevel information
    \item \textbf{\texttt{update-rc.d}} - Install/remove System-V style init links
\end{itemize}

\newpage
\section{Shell Scripting Basics}
\label{sec:scripting}

\subsection{Shell Special Characters}
\begin{itemize}
    \item \textbf{\texttt{|}} - Pipe (redirect output)
    \item \textbf{\texttt{;}} - Command separator
    \item \textbf{\texttt{\&\&}} - Logical AND (run next command if previous succeeds)
    \item \textbf{\texttt{||}} - Logical OR (run next command if previous fails)
    \item \textbf{\texttt{>}} - Output redirection
    \item \textbf{\texttt{>>}} - Append output
    \item \textbf{\texttt{<}} - Input redirection
    \item \textbf{\texttt{2>}} - Error redirection
\end{itemize}

\subsection{Variable Usage}
\begin{lstlisting}[language=bash, caption=Shell variables]
echo $USER       # Current username
echo $HOME       # Home directory
echo $PATH       # Command search path
echo $SHELL      # Current shell
echo $PWD        # Current directory
echo $UID        # User ID
echo $?          # Exit status of last command
\end{lstlisting}

\subsection{Bash Script Example}
\begin{lstlisting}[language=bash, caption=Simple backup script]
#!/bin/bash
# Simple backup script

BACKUP_DIR="/backup"
SOURCE_DIR="/home/user/documents"
DATE=$(date +%Y%m%d)

if [ ! -d "$BACKUP_DIR" ]; then
    mkdir -p "$BACKUP_DIR"
fi

tar -czf "$BACKUP_DIR/backup_$DATE.tar.gz" "$SOURCE_DIR"

if [ $? -eq 0 ]; then
    echo "Backup completed successfully!"
else
    echo "Backup failed!"
fi
\end{lstlisting}

\newpage
\section{Appendix: Quick Reference}
\label{sec:appendix}

\begin{tcolorbox}[colback=red!5!white, colframe=red!75!black, title=\textbf{Essential Commands Cheat Sheet}]
    Most frequently used Linux commands for LPIC-1
\end{tcolorbox}

\subsection{File Operations}
\begin{tabular}{ll}
    \textbf{Command} & \textbf{Description} \\
    \hline
    \texttt{ls -la} & List all files with details \\
    \texttt{cp -r src dst} & Copy recursively \\
    \texttt{mv old new} & Move/rename \\
    \texttt{rm -rf dir} & Remove force recursively \\
    \texttt{find / -name file} & Find files \\
    \texttt{grep pattern file} & Search text \\
\end{tabular}

\subsection{System Monitoring}
\begin{tabular}{ll}
    \textbf{Command} & \textbf{Description} \\
    \hline
    \texttt{top} & Process monitor \\
    \texttt{df -h} & Disk usage \\
    \texttt{free -m} & Memory usage \\
    \texttt{ps aux} & Process list \\
    \texttt{netstat -tulpn} & Network connections \\
\end{tabular}

\subsection{User Management}
\begin{tabular}{ll}
    \textbf{Command} & \textbf{Description} \\
    \hline
    \texttt{useradd username} & Add user \\
    \texttt{passwd username} & Change password \\
    \texttt{usermod -aG group user} & Add user to group \\
    \texttt{userdel -r username} & Delete user with home \\
    \texttt{chmod 755 file} & Change permissions \\
\end{tabular}

\begin{center}
    \rule{0.8\textwidth}{0.5pt}
    \vspace{0.5cm}
    
    \textbf{\Large End of LPIC-1 Study Notes} \\
    \vspace{0.2cm}
    \small Good luck with your certification exam!
\end{center}

\end{document}